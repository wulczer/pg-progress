\documentclass{beamer}

\usepackage[utf8]{inputenc}

\usetheme{CambridgeUS}
\usecolortheme{rose}

\useoutertheme{infolines}
\setbeamertemplate{blocks}[default]
\setbeamertemplate{items}[triangle]
\setbeamertemplate{enumerate items}[square]
\setbeamertemplate{sections/subsections in toc}[square]

\AtBeginSubsection[]
{
  \begin{frame}{Outline}
    \tableofcontents[sectionstyle=show/shaded,subsectionstyle=show/shaded/hide]
  \end{frame}
}

\title{Estimating query progress}
\subtitle{Theory and practice of query progress indication}
\author[Jan Urbański]{Jan Urbański \\ \texttt{j.urbanski@wulczer.org}}
\institute{Ducksboard}
\date[PGCon 2013]{PGCon 2013, Ottawa, May 24}

\begin{document}

\frame{\titlepage}

\begin{frame}[fragile]
  \frametitle{For those following at home}

  \begin{block}{Getting the slides}
    \begin{semiverbatim}
    \$ wget http://wulczer.org/pg-progress.pdf
    \end{semiverbatim}
  \end{block}

  \begin{block}{Trying the code\alert{*}}
    \begin{semiverbatim}
    \$ git clone git://github.com/wulczer/pg-progress.git
    \$ cd pg-progress && make install
    \$ psql -c 'create extension progress'
    \$ ./show-progress.py -c 'table tab1' dbname=progress
    \end{semiverbatim}
  \end{block}

  {\tiny\alert{*} requires modified Postgres from git://github.com/wulczer/postgres/tree/progress-rebasing}
\end{frame}

\begin{frame}
  \tableofcontents
\end{frame}

\section{Problem}
\subsection{Problem statement}

\begin{frame}
\begin{beamercolorbox}[center]{note}
  \Huge Should I kill it yet?
\end{beamercolorbox}
\end{frame}

\begin{frame}
  \frametitle{Usefulness of progress estimation}

  \begin{itemize}
  \item Postgres gives no feedback for running queries
  \item it's hard to decide how much more time will a query take
  \item the decision between cancelling a heavyweight query and letting it
    complete is uninformed
  \end{itemize}
\end{frame}

\begin{frame}
  \frametitle{The devil is in the details}

  \begin{itemize}
  \item while the use case is straightforward, the details are not
  \item what should be an estimator's output?
    \begin{itemize}
    \item ``work done''
    \item percentage completed
    \item time remaining
    \end{itemize}
  \item how should the estimation be relayed to the outside
    \begin{itemize}
    \item the backend running the query is busy... running the query
    \end{itemize}
  \end{itemize}
\end{frame}

\subsection{Qualities of a good estimator}

\begin{frame}
  \frametitle{What makes a good estimator}

  \begin{description}[hardware independence]
  \item[monotonicity] the estimator's value should always grow
  \item[granularity] estimates should vary even small time increments
  \item[low overhead] calculating progress should not be prohibitively
    expensive
  \item[hardware independence] changes in the estimates should not depend on
    how fast the system is
  \end{description}
\end{frame}

\section{Theory}
\subsection{Model of work for query execution}

\begin{frame}[fragile]
  \frametitle{Concepts inside the executor}

  \begin{itemize}
  \item the executor works with a tree structure
  \item the executor tree mirrors the plan tree node for node
  \item each node has a method to produce a tuple or NULL if it's finished
  \item executing the plan is repeatedly fetching a tuple from the root node
  \end{itemize}

  \begin{block}<2->{Executing a query}
    \begin{semiverbatim}
      while (true) \{
          tup = ExecProcNode(root);
          if (tup == NULL)
              break;
          emit(tup);
      \}
    \end{semiverbatim}
  \end{block}
\end{frame}

\begin{frame}
  \frametitle{Individual node execution}

  \begin{itemize}
  \item each node has its own execution method
    \begin{itemize}
    \item for Sequential Scans it fetches the next tuple from the heap
    \item for Materialization it fetches and stores the next tuple from its
      child node
    \item for Sort it stores \alert{all} the tuples from its child node, sorts
      them and returns the top one
    \end{itemize}
  \item nodes can carry additional instrumentation, that gets updated when they
    are executed
  \item each executor node has a link to the correspoding planner node, where
    estimates are kept
  \end{itemize}
\end{frame}

\begin{frame}
  \frametitle{Existing research}

  The basic idea explained here comes from a research paper by Surajit
  Chaudhuri and Vivek Narasayya from Microsoft Research and Ravi Ramamurthy
  from the University of Wisconsin.

  \bigskip

  \begin{thebibliography}{Chaudhuri 2004}

  \bibitem[Chaudhuri 2004]{chaudhuri}
    Chaudhuri, Surajit and Narasayya, Vivek and Ramamurthy, Ravishankar.
    \newblock Estimating progress of SQL queries.
    \newblock {\em SIGMOD '04}, 803--814, 2004.

  \end{thebibliography}
\end{frame}

\begin{frame}
  \frametitle{Model of work}

  \begin{itemize}
  \item query progress is expressed as the number tuples returned from all
    execution nodes
  \item that's called the GetNext() model in the paper
  \item a query is completed when all nodes have finished returning tuples
  \end{itemize}
\end{frame}

\begin{frame}
  \frametitle{Estimator definition}

  \begin{block}{GetNext() model estimator}
    Assume the execution plan has $m$ nodes.

    \bigskip

    For $i = 1..m$, we define \alert<3>{$N_{i}$} as the number of tuples that
    it \alert<3>{will return} and \alert<2>{$K_{i}$} as the number of tuples it
    \alert<2>{has already returned}.

    \bigskip

    The estimator we'll use is
    \begin{equation*}
      gnm = \frac{\sum\limits_{i=1}^{m} \alert<2>{K_{i}}}
      {\sum\limits_{i=1}^{m} \alert<3,4>{N_{i}}}
    \end{equation*}

    \bigskip

    \uncover<4->{The basic challenge is estimating \alert{$N_{i}$}.}
  \end{block}
\end{frame}

\begin{frame}
  \frametitle{Properties of $gnm$}

  \begin{itemize}
  \item kind of, sort of computable in Postgres :)
  \item reasonably straightforward definition
  \item can work off planner estimates, most of the necessary info is already
    being collected
  \item monotonous (usually), fine-grained
  \item but planner estimates can be wrong, especially with joins present
  \end{itemize}
\end{frame}

\subsection{Execution pipelines}

\begin{frame}
  \frametitle{The pipeline concept}

  \begin{block}{Pipeline definition}
    An execution pipeline is the maximal execution plan subtree that executes
    concurrently.
  \end{block}

  \bigskip

  You can think of a pipeline as the set of execution nodes that will be run
  for the pipeline root to produce one tuple.
\end{frame}

\begin{frame}
  \frametitle{Constructing pipelines}

  \begin{itemize}
  \item execution plan leaves start new pipelines
  \item nodes like Sort or Hash start new pipelines
  \item joins combine the pipelines of their children
  \end{itemize}
\end{frame}

\begin{frame}
  \frametitle{Pipeline examples}

  A simple query plan with one pipeline looks like this:
  \begin{center}
    \includegraphics[height=3in]{pipelines-1.pdf}
  \end{center}
\end{frame}

\begin{frame}<1>[label=several-pipelines]
  \frametitle{Pipeline examples cont.}

  A plan with several pipelines:
  \begin{center}
    \includegraphics[height=3in]{pipelines-2.pdf}
  \end{center}
\end{frame}

\begin{frame}
  \frametitle{Pipeline examples cont.}

  Another example:
  \begin{center}
    \includegraphics[height=3in]{pipelines-3.pdf}
  \end{center}
\end{frame}

\subsection{Driver node estimator}

\begin{frame}
  \frametitle{Driver nodes}

  \begin{block}{Driver node definition}
    Driver nodes are leaf nodes of pipelines, except for nodes in the inner
    subtree of a Nested Loops join.
  \end{block}

  \bigskip

  Conceptually, driver nodes are the ones that \alert{drive} the pipeline,
  supplying it with tuples.

  \bigskip

  Because of that, they are better suited to estimate the progress of a
  pipeline than the rest of its nodes.
\end{frame}

\begin{frame}
  \frametitle{Driver node hypothesis}

  \begin{block}{Estimating progress of a running pipeline}
    For pipelines with one driver node that are in progress, we use the
    \alert{driver node hypothesis}, which says that

    \bigskip

    \begin{equation*}
      \frac{\sum\limits_{i=1}^{m} K_{i}}{\sum\limits_{i=1}^{m} N_{i}} \approx \frac{K_{1}}{N_{1}}
    \end{equation*}
  \end{block}
\end{frame}

\begin{frame}
  \frametitle{Unstarted and finished pipelines}

  \begin{block}{Estimating progress of other pipelines}

    For pipelines that are already \alert{finished}, we know
    \begin{equation*}
      \sum\limits_{i=1}^{m} K_{i} = \sum\limits_{i=1}^{m} N_{i}
    \end{equation*}

    For pipelines that are \alert{not yet started}, we use
    \begin{equation*}
      \sum\limits_{i=1}^{m} K_{i} = 0 , \quad \sum\limits_{i=1}^{m} N_{i} = planner\;estimate
    \end{equation*}
  \end{block}
\end{frame}

\begin{frame}
  \frametitle{Final estimator}

  \begin{block}{GetNext() estimator with pipelines}
    For a query comprising $s$ pipelines, the final estimator definition is

    \begin{equation*}
      gnm = \frac{\sum\limits_{i \in P1} K_{i} + ... + \sum\limits_{i \in Ps} K_{i}}
      {\sum\limits_{i \in P1} N_{i} + ... + \sum\limits_{i \in Ps} N_{i}}
    \end{equation*}
  \end{block}
\end{frame}

\section{Practice}
\subsection{Implementation}

\begin{frame}
  \frametitle{Implementation idea}

  The idea was to make the implementation minimally intrusive with regards to
  Postgres code.

  \begin{enumerate}
  \item at executor startup, determine pipelines and driver nodes
  \item when asked to estimate the progress
    \begin{enumerate}
    \item walk all pipelines
    \item calculate $K_{P}$ and $N_{P}$ for each of them
    \end{enumerate}
  \item return the sum of $K_{P}$ divided by the sum of $N_{P}$
  \item use the opportunity to draw a graph of the execution tree and annotate
    it with progress information for each node
  \end{enumerate}
\end{frame}

\begin{frame}[fragile]
  \frametitle{Calculating pipeline states}

\begin{block}{Processing a pipeline}
\begin{semiverbatim}
\uncover<1->{  \alert<1>{k, n} = 0;}
\uncover<1->{  for node in pipeline \{}
\uncover<1->{      k += tuples\_processed(node);}
\uncover<1->{      n += tuples\_estimated(node);}
\uncover<1->{  \}}
\uncover<2->{  if (all(pipeline->driver\_nodes, FINISHED))}
\uncover<2->{      return \alert<2>{k, k};}
\uncover<3->{  else if (any(pipeline->driver\_nodes, STARTED))}
\uncover<3->{      return \alert<3>{k, k * dne(pipeline)};}
\uncover<4->{  else}
\uncover<4->{      return \alert<4>{0, n};}
\end{semiverbatim}
\end{block}
\end{frame}

\begin{frame}
  \frametitle{Communicating with the query backend}

  \begin{itemize}
  \item needed a way to get information from a backend running the query
  \item not trivial, since the backend is busy producing the result
  \item introduce a custom signal handler based on a patch from Simon Riggs
  \item on next CHECK\_FOR\_INTERRUPTS(), calculate progress and put in shared
    memory
  \item signalling backend can then grab results from shmem
  \end{itemize}
\end{frame}

\begin{frame}
  \frametitle{Changes to core Postgres}

  From least to most controversial:

  \begin{itemize}
  \item bulk of the logic is implemented as a Postgres extension
  \item ability to store private data in the main executor structure
  \item hooks for executor node instrumentation
  \item a hook for running user code on SIGUSR1
  \end{itemize}
\end{frame}

\subsection{Hurdles}

\begin{frame}
  \frametitle{Not everything went as smoothly...}

  \begin{itemize}
  \item multiple driver nodes in a pipeline
  \item nodes being executed multiple times
  \item node with more than two children (e.g. Append)
  \item subselect, function scans, recursive queries...
  \item poor planner estimates
  \end{itemize}
\end{frame}

\begin{frame}
  \frametitle{Some solutions}

  \begin{itemize}
  \item for a pipeline with $D$ driver nodes, the estimator is using
    \begin{equation*}
      K = \frac{\sum\limits_{i=1}^{D} K_{i}}{D}, \quad N = \min\limits_{i=1,\dots D} N_{i}
    \end{equation*}
  \item for Nested Loops, the estimated tuple counts of inner child nodes are
    multiplied by estimated outer child tuple count
  \item nodes with more than 2 children are simply assumed blocking and start
    their own pipeline
  \item some nodes are not supported at all (for instance Recursive Union)
  \end{itemize}
\end{frame}

\begin{frame}
  \frametitle{Refining estimates}

  \begin{itemize}
  \item the original paper puts a lot of value in refining estimates as the
    query progresses
  \item each node has a lower and an upper bound for its $N$ value
  \item those bounds are refined as execution progresses:
    \begin{itemize}
    \item Sort or Hash nodes will never increase carditality
    \item when an outer child of a Nested Loops join outputs more tuples than
      expected, recalculate the expected number of loops in the inner subtree
    \item Merge joins will stop as soon as they finish processing one of their
      inputs
    \end{itemize}
  \item a missing feature for now
  \end{itemize}
\end{frame}

\subsection{What next?}

\begin{frame}
  \frametitle{Annoyances}

  \begin{itemize}
  \item correctly identifying pipelines and driver nodes (lots of different
    nodes types to think about)
  \item detecting if a pipeline has started or has finished
    \begin{itemize}
    \item annoyingly tricky
    \item nodes can be executed several times, for example in a Nested Loops
      outer tree
    \end{itemize}
  \item no visibility into a Sort node progress
    \begin{itemize}
    \item important, because Sort nodes are driver nodes
    \end{itemize}
  \item using shared memory to relay estimates to other backends
    \begin{itemize}
    \item haven't even considered the security implications yet...
    \end{itemize}
  \end{itemize}
\end{frame}

\begin{frame}
  \frametitle{Rescanning nodes}

  \begin{itemize}
  \item the executor has a concept of rescanning a node
  \item some nodes are very cheap to rescan, for example Materialization nodes
  \item they will show a large number for tuples returned, but the real work
    was just scanning the child node once
  \item need to consider number of loops manually, since the planner accounts
    for them internally and does not expose the total estimated number of
    tuples
  \item not acconuted for in the original paper at all
  \end{itemize}
\end{frame}

\begin{frame}
  \frametitle{The ``single outer tuple'' problem}

  \begin{itemize}
  \item a common situation is a Nested Loops scan estimating a single outer
    tuple
  \item if the outer child returns more than one tuple, the entire inner
    subtree is rescanned
  \item arguably a planner failure, but because the outer child can end up
    being the only driver node of a large pipeline, estimates are all wrong
  \end{itemize}
\end{frame}

\begin{frame}
  \frametitle{``Single output tuple'' example}

  This tree has only one driver node, which leads to imprecision
  \begin{center}
    \includegraphics[height=3in]{pipelines-1.pdf}
  \end{center}
\end{frame}

\begin{frame}
  \frametitle{MultiExecProcNode}

  \begin{itemize}
  \item some nodes are ``special'' and don't follow the $GetNext()$ model
    \begin{itemize}
    \item Hash nodes build the entire hashtable in one go
    \item Bitmap index scans scan the entire index when building the bitmap
    \end{itemize}
  \item if they'd be just blocking, it wouldn't have been a problem
  \item being outside of $GetNext()$ means no visibility into their progress
  \item particularily bad when they're driver nodes
  \end{itemize}
\end{frame}

\begin{frame}
  \frametitle{Utility commands}

  \begin{itemize}
  \item this model does not allow estimating the progress of utility commands
    in any way
  \item things like COPY, VACUUM, CREATE INDEX, ANALYSE
  \item a shame, since these are just the kinds of statements that often would
    benefit from estimation
  \item possibly a hybrid approach would work, special-casing utility
    statements
  \end{itemize}
\end{frame}

\begin{frame}
  \frametitle{Current state}

  \begin{itemize}
  \item requires small changes to core Postgres, but some are controversial
  \item gives some useful information at least
  \item the driver node estimator looks like an interesting metric to use
  \item could be reimplemented as returning just raw execution data and let
    other tools calculate a single progress value
  \item graphical dumps are more a debugging tool, should probably use
    EXPLAIN-ish format and external tools could transform it into images
  \item need to implement estimate refining
  \item for the time being, a proof of concept
  \end{itemize}
\end{frame}

\subsection*{Questions}

\begin{frame}
\begin{beamercolorbox}[center]{note}
  \Huge Questions?
\end{beamercolorbox}
\end{frame}

\end{document}
